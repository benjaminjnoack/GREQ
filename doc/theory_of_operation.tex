\documentclass[12pt,a4paper]{article}

%opening
\title{Theory of Operation}
\author{Benjamin J. Noack}
\begin{document}

\maketitle
\begin{abstract}
	How Things Work
\end{abstract}

\section{Electronics}

\subsection{Pin Connections}

\begin{tabular}{| c| c| c|}
	\hline
	J1-2 & PTC3 & MSGEQ7 Reset \\
	\hline
	J1-2 & PTC4 & MSGEQ7 Strobe \\
	\hline
	J1-14 & ADC0-4 & MSGEQ7 Out \\
	\hline
	J2-2 & PTB18 & Encoder Phase A \\
	\hline
	J2-4 & PTB19 & Encoder Phase B \\
	\hline
	J2-6 & PTD0 & Encoder Button \\
	\hline
	J2-13 & I2C0-SCL & HT16K33 SCL \\
	\hline
	J2-15 & I2C0-SDA & HT16K33 SDA \\
	\hline
	J4-2 & ADC1-10 & Equalizer Band 0 \\
	\hline
	J4-4 & ADC1-11 & Equalizer Band 1 \\
	\hline
	J4-6 & ADC1-12 & Equalizer Band 2 \\
	\hline
	J4-8 & ADC1-13 & Equalizer Band 3 \\
	\hline
	J4-7 & ADC1-16 & Equalizer Band 4 \\
	\hline
	J6-4 & SPI1-PCS0 & MAX7219 CS \\
	\hline
	J6-6 & SPI1-SOUT & MAX7219 MOSI \\
	\hline
	J6-5 & SPI1-SCK & MAX7219 SCK \\
	\hline
\end{tabular}

\appendix
\section{Timer Configuration}

The BARS task is designed to run at a fixed frequency of 60Hz (or a period of 16.6 ms).
One of the K66's Flex Timer modules is used as a simple timer.
An interrupt will fire when the timer overflows.
The ISR will send a signal releasing the task.

The timer's CNT (count) register is 16 bits.
Therefore, the maximum value the counter can hold is 0xFFFF (65,535).
The timer "ticks" at the frequency of the supplied clock.
The supplied bus clock has a frequency of 60MHz.
Therefore, the period of a tick is \(\frac{1}{60000000} = 1.6 \times 10^{-8}\).
Even if the maximum value (0xFFFF) were used as the MOD (modulo) value for the counter,
it would overflow every \(\frac{1}{60000000} \times 65353 = 0.00109225\),
or roughly 1.09 ms. \(\frac{1}{0.00109225} = 915.54\) or roughly 915Hz. Way too fast.

The solution is to divide the timer's clock signal.
Various powers of two are available as clock dividers.
If the clock is divided by 16 then the frequency is \(\frac{60000000}{16} = 3750000\)
or 3.75MHz.
The period of the clock is then \(\frac{1}{3750000} = 2.66 \times 10^{-7}\).
It is then a matter of determining the correct modulo value to achieve 60Hz overflow frequency.
The problem may be expressed generally as the following.
Where \(P_t\) is the target period, \(P_c\) is the input clock frequency, and \(t\) is the requisite ticks:

\[P_t = P_c \times t \]

Substituing a target frequency of 60Hz, and an input clock of 3.75MHz,

\[ \frac{1}{60} = \frac{1}{3750000} \times t\]
\[ \frac{\frac{1}{60}}{\frac{1}{3750000}} = t\]
\[ \frac{3750000}{60} = t \]
\[ 62500 = t \]

To summarize, the timer module's input clock must be divided in order to provide a suitable time base.
Once divided, a modulo of 62500 will cause an overrun every 16.6 ms or 60Hz.

\end{document}

